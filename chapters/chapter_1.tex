\section{Alicja Solarz}


\begin{figure}[htbp] 
    \centering
   \includegraphics[width=1\textwidth]{pictures/postaw_na_milion.jpg}
    \caption{Usmiechniety prowadzacy :)}
    \label{fig:postaw na milion!}
\end{figure}

\begin{table}[htbp]
\centering
\begin{tabular}{||c c c c||} 
 \hline
 Kol1 & Kol2 & Kol2 & Kol3 \\ [0.5ex] 
 \hline\hline
 1 & 2 & 3 & 4 \\ 
 \hline
 5 & 6 & 7 & 8 \\
 \hline
 9 & 10 & 11 & 12 \\
 \hline
\end{tabular}
\label{tab:ala}
\caption{Fajna tabelka}
\end{table}

Pythagorean theorem:
\[ a^2 + b^2 = c^2 \]

\begin{itemize}
  \item[!]  Mr. and Mrs. Dursley, of number four, Privet Drive, were
proud to say that they were perfectly normal, thank
you very much.
  \item[!]  They were the last people you’d expect to be involved in anything strange or mysterious, because they just didn’t hold with such nonsense.
  \item[!] Mr. Dursley was the director of a firm called Grunnings, which
made drills.
\end{itemize}

\begin{enumerate}
  \item  He was a big, beefy man with hardly any neck, although he did have a very large mustache. 
  \item Mrs. Dursley was thin
and blonde and had nearly twice the usual amount of neck, which
came in very useful as she spent so much of her time craning over
garden fences, spying on the neighbors.
  \item The Dursleys had a small
son called Dudley and in their opinion there was no finer boy
anywhere.

\end{enumerate}

\textbf{Beekeepers} use smoke to calm bees when they are collecting honey or relocating a hive. Bees make honey to feed their young and so they have something to eat during the winter.\par
Bees are known as \underline{"pollinators"}, meaning they help plants live and reproduce by transferring pollen between various species of flowering plants like flowers. They carry pollen on their legs and body from one flower to another, helping to create what's known as "genetic variety" by distributing different genes throughout many plants and flowers.

 \ref{fig:postaw na milion!}
 odwolanie